\documentclass{beamer}
\usetheme{m}  % Use metropolis theme
\title{Codierungstheorie - Praktika 2}
\date{\today}
\author{Wombacher Sascha \newline}
\usepackage{float}
\usepackage{listings}
\usepackage[ngerman]{babel}

\lstset{
	language=C++,
    tabsize=4,
    keepspaces,
    extendedchars=true,
    rulecolor=\color{black},
    basicstyle=\footnotesize,
    aboveskip=5pt,
    upquote=true,
    columns=fixed,
    showstringspaces=false,
    extendedchars=true,
    breaklines=true,
    frame=single,
    showtabs=true,
    showspaces=false,
    showstringspaces=false,
    basicstyle=\tiny,
    keywordstyle=\color{blue}
}

\begin{document}

  \lstdefinestyle{numbers}
  {numbers=left, stepnumber=1, numberstyle=\tiny, numbersep=10pt}
  \lstdefinestyle{nonumbers}
  {numbers=none}


  \maketitle
  \begin{frame}{Agenda}
    \setbeamertemplate{section in toc}[sections numbered]
    \tableofcontents[hideallsubsections]
  \end{frame}

  \section{Datenstrukturwahl}
  \begin{frame}{Datenstruktur \newline - std::Array im Vergleich zu std::Vector}
  	\begin{enumerate}
	  	\item Als Grunddatenstruktur wurde ein std::Array gewählt
	  	\item Nachteile:
	  	\begin{enumerate}
		  	\item Feste Arraylänge ist für größer werdende Polynome ungeeignet
		  	\item Die Abfrage des Grades benötigt im worst-case \newline O(N), N = ArrayLänge
		  	\item Höherer Speicherverbrauch (viele 0en)
		  	\item Tendenziell langsamer
	  	\end{enumerate}
	  	\item Vorteile:
	  	\begin{enumerate}
		  	\item Erhöhung des Grades erstellt kein zweites Array \newline (verglichen mit std::vector, worst case)
		  	\item Jeder nimmt std::vector => mal was anderes :D
		  	\item Bischen mehr template programming :)
	  	\end{enumerate}
	\end{enumerate}
  \end{frame}
  
  \section{Modulare Polynommultiplikation}
    \begin{frame}{Modulare Polynommultiplikation \newline - Algorithmen Beschreibung}
		Bekannte Parameter: Poly1, Poly2, DivPoly, PolyBasis
    	\begin{enumerate}
  	  	\item Berechne: Poly1 * Poly2, in temporäres Array (Tmp)
  	  	\item Reduziere alle Komponenten von Tmp zur PolyBasis
  	  	\item Solange Tmp.Grad >= DivPoly.Grad
  	  	\begin{enumerate}
	  	  	\item Berechne Additions-Faktor F, mit F = PolyBasis - Tmp.Grad \newline (gilt nur wenn DivPoly[DivPoly.Grad] == 1 ist)
	  	  	\item Berechne Shift S, mit S = Tmp.Grad - DivPoly.Grad
	  	  	\item Addiere Komponentenweise das "geshiftete" DivPoly zu Tmp
	  	  	\item Reduziere alle Komponenten von Tmp zur PolyBasis
  	  	\end{enumerate}
  	  	\item gib Tmp zurück
  	  	\end{enumerate}
    \end{frame}
    
    \begin{frame}[fragile]{Modulare Polynommultiplikation \newline - Programmcode}
    	\begin{lstlisting}[style=numbers]
// create temporary polynom with 2*_MaxLength
// this is the maximum grad the *-Operation can create
Polynom<_MaxDegree * 2, _BaseValue> tmp, tmpResult;

// calculate "multiplication" into tmp poly
for (int i = 0, myDegree = this->degree(), polyDegree = poly.degree(); i <= myDegree; ++i){
    for (int k = 0; k <= polyDegree; ++k)
        tmpResult.m_Data[i + k] += poly.m_Data[k] * this->m_Data[i];
}
tmpResult._truncateToBaseValue();
const int divisionDegree = DIVISION_POLYNOM->degree();
for (int tmpResDegree = tmpResult.degree(); tmpResDegree >= divisionDegree; tmpResDegree = tmpResult.degree()){
    int factor = _BaseValue - tmpResult.m_Data[tmpResDegree];
    int shift = tmpResDegree - divisionDegree;

    for (int i = 0; i <= divisionDegree; ++i)
        tmpResult.m_Data[shift + i] += factor * DIVISION_POLYNOM->m_Data[i];
    tmpResult._truncateToBaseValue();
}
MY_TYPE toReturn;
for (int i = 0; i < _MaxDegree; ++i)
    toReturn.m_Data[i] = tmpResult.m_Data[i];

// return "shrinked" result
return toReturn;
\end{lstlisting}
\end{frame}

    \begin{frame}{Modulare Polynommultiplikation - \newline Assoziativgesetz Überprüfung, Seite 1}
	    Für diese Aufgabe werden irreduzieblen Polynome aus Aufgabe 4 verwendet
    	\begin{itemize}
  	  	\item Überführe eine Ganzzahl in ein Polynom
  	  	\begin{enumerate}
	  	  	\item Erstelle einen Iterator $i = 0$
	  	  	\item Schreibe an Arrayposition $i$ das Modulo der Ganzzahl mit der Polynombasis \label{IterStart}
	  	  	\item Dividiere die Ganzzahl durch die Polynombasis
	  	  	\item Inkrementiere $i$ \label{IterEnd}
	  	  	\item Wiederhole Schritt \ref*{IterStart} - \ref*{IterEnd} solange Ganzzahl > 0 gilt
  	  	\end{enumerate}
  	  	\end{itemize}
    \end{frame}
    
    \begin{frame}{Modulare Polynommultiplikation - \newline Assoziativgesetz Überprüfung, Seite 2}
		Iteriere über drei (unabhängige) Iteratoren \newline  $iter1, iter2, iter3 \epsilon [0, TableSize - 1]$
      	\begin{enumerate}
	    	\item Konvertiere Iterator $iter1$ in Polynom \textit{Poly1}
	    	\item Konvertiere Iterator $iter2$ in Polynom \textit{Poly2}
	    	\item Konvertiere Iterator $iter3$ in Polynom \textit{Poly3}
	    	\item Erstelle $Poly4 = (Poly1 *  Poly2) * Poly3$
	    	\item Erstelle $Poly5 =  Poly1 * (Poly2  * Poly3)$
	    	\item Wenn $Poly4 \neq Poly5$ gilt, wirf Exception
      	\end{enumerate}
   \end{frame}
    
    \begin{frame}[fragile]{Modulare Polynommultiplikation - \newline Assoziativgesetz Programmcode}
    	\begin{lstlisting}[style=numbers]
Polynom<_ArraySize, _BaseValue>::DIVISION_POLYNOM = &div; // set division poly
auto generatePolynom = [](int value) -> Polynom<_ArraySize, _BaseValue>{
    Polynom<_ArraySize, _BaseValue> toReturn;
    for (int i = 0; value; ++i){
        toReturn.m_Data.at(i) = value % _BaseValue;
        value /= _BaseValue;
    }
    return toReturn;
};
// check for error
for (int iter1 = 0; iter1 < _TestLength; ++iter1){
    auto poly1 = generatePolynom(iter1);

    for (int iter2 = 0; iter2 < _TestLength; ++iter2){
        auto poly2 = generatePolynom(iter2);

        for (int iter3 = 0; iter3 < _TestLength; ++iter3){
            auto poly3 = generatePolynom(iter3);
            
            auto poly4 = (poly1 * poly2) * poly3;
            auto poly5 = poly1 * (poly2  * poly3);
            if (poly4 != poly5)
                throw std::runtime_error("Error: wrong implementation");
        }
    }
}
std::cout << "\rmultiplication found no errors" << std::endl;
\end{lstlisting}
\end{frame}

	\section{Tabellengenerierung}
    \begin{frame}{Tabellengenerierung - \newline Beschreibung}
    	Iteriere über zwei (unabhängige) Iteratoren \newline $iter1, iter2 \epsilon [0, TableSize - 1]$
    	\begin{enumerate}
  	  	\item Konvertiere Iterator $iter1$ in Polynom \textit{Poly1}
  	  	\item Konvertiere Iterator $iter2$ in Polynom \textit{Poly2}
  	  	\item Gib $poly1 * poly2$ aus
  	  	\end{enumerate}
    \end{frame}
    
    \begin{frame}[fragile]{Tabellengenerierung - \newline Programmcode}
	    Anmerkung: Teile des Programmcodes (Tabllenlienien) \newline wurden entfert (ansonsten genügt eine Folie nicht)
    	\begin{lstlisting}[style=numbers]
// generate polynoms from int-iterator
auto generatePolynom = [](int value) -> Polynom<_PolySize, _BaseValue>{
    Polynom<_PolySize, _BaseValue> toReturn;
    for (int i = 0; value; ++i){
        toReturn.m_Data[i] = value % _BaseValue;
        value /= _BaseValue;
    }
    return toReturn;
};
Polynom<_PolySize, _BaseValue>::DIVISION_POLYNOM = &divisionPoly;

// table itself
for (int iter1 = 0; iter1 < tableSize; ++iter1){
    auto poly1 = generatePolynom(iter1);

    for (int iter2 = 0; iter2 < tableSize; ++iter2){
        auto poly2 = generatePolynom(iter2);

        // foo is a function which perforams addition or multiplication
        std::cout << foo(poly1, poly2);
    }
}
\end{lstlisting}
\end{frame}


\section*{Vielen Dank für die Aufmerksamkeit!}
  
\end{document}
