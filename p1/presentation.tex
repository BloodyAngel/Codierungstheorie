\documentclass{beamer}
\usetheme{m}  % Use metropolis theme
\title{Codierungstheorie - Praktika 1}
\date{\today}
\author{Wombacher Sascha \newline}
\usepackage{float}
\usepackage{listings}
\usepackage[ngerman]{babel}
%\institute{Centre for Modern Beamer Themes}

\lstset{
	language=C++,
    tabsize=4,
    keepspaces,
    extendedchars=true,
    rulecolor=\color{black},
    basicstyle=\footnotesize,
    aboveskip=5pt,
    upquote=true,
    columns=fixed,
    showstringspaces=false,
    extendedchars=true,
    breaklines=true,
    frame=single,
    showtabs=true,
    showspaces=false,
    showstringspaces=false,
    basicstyle=\tiny,
    keywordstyle=\color{blue}
}

\begin{document}

  \lstdefinestyle{numbers}
  {numbers=left, stepnumber=1, numberstyle=\tiny, numbersep=10pt}
  \lstdefinestyle{nonumbers}
  {numbers=none}


  \maketitle
  \begin{frame}{Agenda}
    \setbeamertemplate{section in toc}[sections numbered]
    \tableofcontents[hideallsubsections]
  \end{frame}

  \section{Generierung des Huffman-Baums}
  \begin{frame}{Generierung des Huffman-Baums - \newline Wahrscheinlichkeit eines Zeichen}
  	\begin{enumerate}
	  	\item für jedes Zeichen $z$ der Eingabezeichenkette wird die Häufigkeit (Anzahl) errechnet
	  	\item Anschließend wird diese Anzahl durch die Eingabelänge Normiert (Intervall zwischen (0, 1])
	\end{enumerate}
  \end{frame}
  
    \begin{frame}[fragile]{Generierung des Huffman-Baums - \newline Wahrscheinlichkeit - Programmcode}
    	\begin{lstlisting}[style=numbers]
  float numElements = 0;
  for (auto& e : containerType){
      bool elementFound = false;
      for (auto& iter : elementCounter){
          if (iter->data.first == e){
              elementFound = true;
              ++iter->data.second;
              break;
          }
      }
      if (!elementFound){
          elementCounter.reserve(1);
          elementCounter.push_back(std::make_unique<LEAVE>(LEAVE(e, 1.f)));
      }
      ++numElements;
  }
  elementCounter.shrink_to_fit();
  for (auto& e : elementCounter)
      e->data.second /= numElements;
\end{lstlisting}
\end{frame}
    
  
  \begin{frame}{Generierung des Huffman-Baums - \newline Baumgenerierung}
  	\begin{enumerate}
  		\item Erstelle je ein Node pro Zeichen
  		\item Sortiere alle Nodes aufsteigend anhand ihrer Wahrscheinlichkeit
  		\item Verbinde die zwei Nodes mit den geriengsten Wahrscheinlichkeiten
  		\item Füge das entstehende Node in die Zeichenliste ein \newline (Wahrscheinlichkeit = Summe der einzel Nodes)
  		\item Stelle die Sortierung wieder her \newline (\textit{insertion sort} - Ansatz)
  		\item Wiederhole Schritt 3-5 bis nur noch ein Node existiert \newline (=> Tree-Root)
  	\end{enumerate}
  \end{frame}
  
     \begin{frame}[fragile]{Generierung des Huffman-Baums - \newline Baumgenerierung - Programmcode}
     	\begin{lstlisting}[style=numbers]
// sort elements
std::sort(std::begin(elementCounter), std::end(elementCounter), [](const auto& element1, const auto& element2) -> bool{
    // data.second = symbol probability
    if (element1->data.second < element2->data.second)
        return true;
    return false;
});
\end{lstlisting}
\end{frame}
  
     \begin{frame}[fragile]{Generierung des Huffman-Baums - \newline Baumgenerierung - Programmcode}
     	\begin{lstlisting}[style=numbers]
// create huffman tree
for (std::size_t i = 0; i < elementCounter.size() - 1; ++i){
    // pick first 2 elements and stick them together
    // and stick them into the second element -> last element will be the tree's root
    std::unique_ptr<NODE>& ptr1 = elementCounter[i];
    std::unique_ptr<NODE>& ptr2 = elementCounter[i + 1];

    std::unique_ptr<NODE> ptr = std::make_unique<NODE>(ptr1->data.first, ptr1->data.second + ptr2->data.second);
    ptr-> leftNode = std::move(ptr1);
    ptr->rightNode = std::move(ptr2);

    ptr2 = std::move(ptr);
    ptr2->leftNode ->parentNode = ptr2.get();
    ptr2->rightNode->parentNode = ptr2.get();

    // sort elements again, using the "insertion sort" appraoch
    for (int iter = i + 1; iter < elementCounter.size() - 1; ++iter){
        auto& e1 = elementCounter[iter];
        auto& e2 = elementCounter[iter + 1];

        if (e1->data.second > e2->data.second)
            std::swap(e1, e2);
        else
            break;
    }
}
\end{lstlisting}
\end{frame}

  
  \begin{frame}{Generierung des Huffman-Baums - \newline Speicherung} \label{HeaderFile}
	für die Weiterverarbeitung ist das Speichern des Huffman-Baums ein wichtiger Bestandteil
  	\begin{enumerate}
  		\item Ersten 32Bit beschreiben die Anzahl der Zeichen im Baum (Blätter)
  		\item Die Folgenden 8Bit beschreiben die Länge eines Zeichens \newline (Bsp.: char = 1, int32 = 4)
  		\item Nehme ein Zeichen aus Huffman-Baum
  		\item Schreibe die Gesamtlänge des Zeichen + Codierung in die Folgenden 8Bit
  		\item Schreibe das Zeichen in die Nächsten Bits
  		\item Schreibe die Codierung des Zeichen in die folge Bits
  		\item Wiederhole Schritt 3-6 für jedes Zeichen im Huffman-Baum
  	\end{enumerate}
  \end{frame}
  
  \begin{frame}[fragile]{Generierung des Huffman-Baums - \newline Speicherung - Programmcode}
  	\begin{lstlisting}[style=numbers]
BitWriter<> writer(output);
const char* count = reinterpret_cast<const char*>(&leaveCount);
writer.addByte(count[0]);
writer.addByte(count[1]);
writer.addByte(count[2]);
writer.addByte(count[3]);

for (const _Leave<T>* ptr = this->m_FirstLeave; ptr; ptr = ptr->nextLeave){
     writer.addByte(8 * (1 + (char)sizeof(T)) + (char)ptr->m_Coding.size());

     // data: first = symbol
     const char* tmpPtr = reinterpret_cast<const char*>(&ptr->data.first);
     
     for (int i = 0; i < sizeof(T); ++i)
          writer.addByte(tmpPtr[i]);

     writer.addBits(ptr->m_Coding);
}
writer.flush();
\end{lstlisting}
\end{frame}
    
  \section{Zeichencodierung}
  \begin{frame}{Zeichencodierung}
  	\begin{enumerate}
  		\item Generiere pro Zeichen die jeweilige Codierung \newline (von Root gesehen: leftNode = 1, rightNode = 0)
  		\item Lese ein Zeichen $z$ des Input-Strings
  		\item Finde die Codierung für $z$
  		\item Füge die Codierung dem BitWriter hinzu
  		\item Wiederhole Schritte 2-4 für alle Zeichen des Strings
  		\item Flush für den BitWriter
  	\end{enumerate}
  \end{frame}
  
  \begin{frame}[fragile]{Zeichencodierung - \newline Programmcode Teil 1}
  	\begin{lstlisting}[style=numbers]
// rucursive call starting at tree root
bool isLeave = true;
if (this->leftNode){
    isLeave = false;
    this->leftNode->generateCodings();
}
if (this->rightNode){
    isLeave = false;
    this->rightNode->generateCodings();
}
if (isLeave){
    _Leave<T2>* THIS = static_cast<_Leave<T2>*>(this);
    std::vector<bool>& coding = THIS->m_Coding;
    for (const _Node<T2>* currentPtr = this->parentNode, *previousePtr = this;
         currentPtr;
         currentPtr  = currentPtr->parentNode)
    {
         // 1 coding left, 0 right
         if (currentPtr->leftNode.get() == previousePtr)
             coding.push_back(1);
         else
             coding.push_back(0);
         previousePtr = currentPtr;
    }
    // vector is generated by bottom up approach -> inverse its order for top down
    std::reverse(std::begin(coding), std::end(coding));
}
\end{lstlisting}
\end{frame} 
  
  \begin{frame}[fragile]{Zeichencodierung - \newline Programmcode Teil 2}
  	\begin{lstlisting}[style=numbers]
BitWriter<> writer(output);
// generate coding table
std::array<const _Leave<T>*, 256> table;
for (auto& e :table)
     e = nullptr;

for (const _Leave<T>* iter = this->m_FirstLeave; iter; iter = iter->nextLeave)
    table[iter->data.first] = iter;
// write symbols
for (const auto& e : str){
    if (!table[e])
        throw "Error, symbol not found in huffmantree";

    writer.addBits(table[e]->m_Coding);
}
writer.flush();
\end{lstlisting}
\end{frame} 
  
  \section{Rekunstruktion des Huffman-Baums}
  \begin{frame}{Rekunstruktion des Huffman-Baums}
  	\begin{enumerate}
  		\item Öffne die erstellte Header-Datei, siehe Folie \textit{\ref{HeaderFile}}
  		\item Lese \textit{invertert} wie zuvor beschrieben
  	\end{enumerate}
  \end{frame}
  
  
  \section{Lesen des codierten Strings}
  \begin{frame}{Lesen des codierten Strings}
  	\begin{enumerate}
  		\item Setze Pointer auf Root
  		\item Lese ein Bit des Inputstreams (BitReader)
  		\item Verfolge Pointer anhand des Bits \newline (1: Pointer = Pointer->left, 0: Pointer = Pointer->right)
  		\item Zeigt der Pointer auf ein Blatt?
  		\begin{itemize}
  			\item Wenn Ja : Schreibe Zeichen, setze Pointer auf Root
  		\end{itemize}
  		\item Wiederhole Schritt 2 bis 4 für jedes Bit des Inputstreams
  	\end{enumerate}
  \end{frame}
  
  \begin{frame}[fragile]{Lesen des codierten Strings - \newline Programmcode} \label{EntropieLesen}
  	\begin{lstlisting}[style=numbers]
std::fstream input(filename, std::fstream::in);
if (!input)
    throw "Error, file can't be opened for reading";

float bitCount = 0.f;
float characterCount = 0.f;

BitReader<> reader(input);
while(reader.good()){
    std::unique_ptr<_Node<T>>* iter = &this->m_Root;

    // find symbol
    while((*iter)->leftNode || (*iter)->rightNode){
        bool left = reader.readBit();
        ++bitCount;

        if (left)
            iter = &(*iter)->leftNode;
        else
            iter = &(*iter)->rightNode;
    }
    container.push_back((*iter)->data.first);
    ++characterCount;
}
std::cout << "Newly calculated entropie: " << bitCount / characterCount << std::endl;
\end{lstlisting}
\end{frame}  

\section{Entropie}
\begin{frame}{Entropie}
Die Entropie wird auf zwei wegen errechnet und verglichen
\begin{itemize}
	\item Wie auf der Folie \ref*{EntropieLesen} zu erkennen
	\begin{itemize}
	\item Zähle die verwendeten bits
	\item Dividire diese durch die Anzahl der erhaltenen Zeichen
	\end{itemize}
	
	\item Errechnung nach der Huffman-Baum Generierung
	\begin{itemize}
	\item Berechne die Einzelwahrscheinlichkeiten jedes Zeichens
	\item Multipliziere dies mit der Codierlänge des Zeichens
	\item Die Aufaddierung dieser entspricht die Entropie
	\end{itemize}
\end{itemize}
\end{frame}

\section*{Vielen Dank für die Aufmerksamkeit!}
  
\end{document}
