\documentclass{beamer}
\usetheme{m}  % Use metropolis theme
\title{Codierungstheorie - Praktika 1}
\date{\today}
\author{Staubach Jan \newline Wombacher Sascha \newline}
\usepackage{float}
\usepackage{listings}
\usepackage[ngerman]{babel}
%\institute{Centre for Modern Beamer Themes}

    \lstset{
    	language=C++,
    	tabsize=4,
    	keepspaces,
    	extendedchars=true,
    	rulecolor=\color{black},
    	basicstyle=\footnotesize,
    	aboveskip=5pt,
    	upquote=true,
    	columns=fixed,
    	showstringspaces=false,
    	extendedchars=true,
    	breaklines=true,
    	frame=single,
    	showtabs=true,
    	showspaces=false,
    	showstringspaces=false,
%    	escapeinside={\%*}{\*)},
    }

\begin{document}
  \maketitle
  \begin{frame}{Agenda}
    \setbeamertemplate{section in toc}[sections numbered]
    \tableofcontents[hideallsubsections]
  \end{frame}

  \section{Generierung des Huffman-Baums}
  \begin{frame}{Generierung des Huffman-Baums - \newline Wahrscheinlichkeit eines Zeichen}
  	\begin{enumerate}
	  	\item für jedes Zeichen $z$ der Eingabezeichenkette wird die Häufigkeit (Anzahl) errechnet
	  	\item Anschließend wird diese Anzahl durch die Eingabelänge Normiert (Intervall zwischen (0, 1])
	\end{enumerate}
  \end{frame}
  
  \begin{frame}{Generierung des Huffman-Baums - \newline Baumgenerierung}
  	\begin{enumerate}
  		\item Erstelle je ein Node pro Zeichen
  		\item Sortiere alle Nodes aufsteigend anhand ihrer Wahrscheinlichkeit
  		\item Verbinde die zwei Nodes mit den geriengsten Wahrscheinlichkeiten
  		\item Füge das entstehende Node in die Zeichenliste ein \newline (Wahrscheinlichkeit = Summe der einzel Nodes)
  		\item Stelle die Sortierung wieder her \newline (\textit{insertion sort} - Ansatz)
  		\item Wiederhole Schritt 3-5 bis nur noch ein Node existiert \newline (=> Tree-Root)
  	\end{enumerate}
  \end{frame}
  
  \begin{frame}{Generierung des Huffman-Baums - \newline Speicherung} \label{HeaderFile}
	für die Weiterverarbeitung ist das Speichern des Huffman-Baums ein wichtiger Bestandteil
  	\begin{enumerate}
  		\item Ersten 32Bit beschreiben die Anzahl der Zeichen im Baum (Blätter)
  		\item Die Folgenden 8Bit beschreiben die Länge eines Zeichens \newline (Bsp.: char = 1, int32 = 4)
  		\item Nehme ein Zeichen aus Huffman-Baum
  		\item Schreibe die Gesamtlänge des Zeichen + Codierung in die Folgenden 8Bit
  		\item Schreibe das Zeichen in die Nächsten Bits
  		\item Schreibe die Codierung des Zeichen in die folge Bits
  		\item Wiederhole Schritt 3-6 für jedes Zeichen im Huffman-Baum
  	\end{enumerate}
  \end{frame}
  
  \lstdefinestyle{numbers}
  {numbers=left, stepnumber=1, numberstyle=\tiny, numbersep=10pt}
  \lstdefinestyle{nonumbers}
  {numbers=none}
  
  \begin{frame}[fragile]{Generierung des Huffman-Baums - \newline Speicherung - Programmcode}
  	\begin{lstlisting}[style=numbers]
BitWriter<> writer(output);
writer.add(leaveCount);

for (const _Leave<T>* ptr = this->m_FirstLeave; ptr; ptr = ptr->nextLeave){
writer.addByte(8 * (1 + (char)sizeof(T)) + (char)ptr->m_Coding.size());
const char* tmpPtr = reinterpret_cast<const char*>(&ptr->data.first);
for (int i = 0; i < sizeof(T); ++i)
writer.addByte(tmpPtr[i]);
writer.addBits(ptr->m_Coding);
}
writer.flush();
    \end{lstlisting}
\end{frame}
    
  \section{Zeichencodierung}
  \begin{frame}{Zeichencodierung}
  	\begin{enumerate}
  		\item Generiere pro Zeichen die jeweilige Codierung \newline (von Root gesehen: leftNode = 1, rightNode = 0)
  		\item Lese ein Zeichen $z$ des Input-Strings
  		\item Finde die Codierung für $z$
  		\item Füge die Codierung dem BitWriter hinzu
  		\item Wiederhole Schritte 2-4 für alle Zeichen des Strings
  		\item Flush für den BitWriter
  	\end{enumerate}
  \end{frame}
  
  
  
  \section{Rekunstruktion des Huffman-Baums}
  \begin{frame}{Rekunstruktion des Huffman-Baums}
  	\begin{enumerate}
  		\item Öffne die erstellte Header-Datei, siehe Folie \textit{\ref{HeaderFile}}
  		\item Lese \textit{invertert} wie zuvor beschrieben
  	\end{enumerate}
  \end{frame}
  
  
  \section{Lesen des codierten Strings}
  \begin{frame}{Lesen des codierten Strings}
  	\begin{enumerate}
  		\item Setze Pointer auf Root
  		\item Lese ein Bit des Inputstreams (BitReader)
  		\item Verfolge Pointer anhand des Bits \newline (1: Pointer = Pointer->left, 0: Pointer = Pointer->right)
  		\item Zeigt der Pointer auf ein Blatt?
  		\begin{itemize}
  			\item Ja  : Schreibe Zeichen, setze Pointer auf Root
  		\end{itemize}
  		\item Wiederhole Schritt 2 bis 4 für jedes Bit des Inputstreams
  	\end{enumerate}
  \end{frame}
  
  

\section*{Vielen Dank für die Aufmerksamkeit!}
  
\end{document}
