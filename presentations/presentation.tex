\documentclass{beamer}
\usetheme{m}  % Use metropolis theme
\title{Codierungstheorie - Praktika 3}
\date{\today}
\author{Wombacher Sascha \newline}
\usepackage{float}
\usepackage{listings}
\usepackage[utf8]{inputenc}
\usepackage[ngerman]{babel}



\lstset{
	language=C++,
    tabsize=4,
    keepspaces,
    extendedchars=true,
    rulecolor=\color{black},
    basicstyle=\footnotesize,
    aboveskip=5pt,
    upquote=true,
    columns=fixed,
    showstringspaces=false,
    extendedchars=true,
    breaklines=true,
    frame=single,
%    showtabs=true,
    showspaces=false,
    showstringspaces=false,
    basicstyle=\tiny,
    keywordstyle=\color{blue}
}

\begin{document}

  \lstdefinestyle{numbers}
  {numbers=left, stepnumber=1, numberstyle=\tiny, numbersep=10pt}
  \lstdefinestyle{nonumbers}
  {numbers=none}


  \maketitle
  \begin{frame}{Agenda}
    \setbeamertemplate{section in toc}[sections numbered]
    \tableofcontents[hideallsubsections]
  \end{frame}
  
  
  \section{Generatormatrix Allgemein}
    \begin{frame}{Generatormatrix}
	\begin{itemize}
		\item Sind $k \times n$ Matrizen
		\item Sind Matrizen mit linear unabhängigen Zeilen
		\item Werden verwendet um eine Nachricht der länge $k$ in ein Codewort der länge $n$ zu überführen (Matrixmultiplikation)
		\item Jede Zeile ist auch ein Codewort
  	\end{itemize}
    \end{frame}

  \section{Kanonische Generatormatrix}
  \begin{frame}{Kanonische Generatormatrix}
  	\begin{itemize}
	  	\item Verwende Gauß zum "Lösen" \space der Generatormatrix
	  	\item Anpassungen an Gauß
	  	\begin{itemize}
	  	\item Wird eine 0-Zeile gefunden wirf Exception \newline (Zeilen sind nicht linear unabhängig)
	  	\item Diagonale 1en werden erzwungen \newline (Spalten werden notfalls getauscht)
	  	\end{itemize}
	\end{itemize}
  \end{frame}
  
  \section{Kontrollmatrix}
  \begin{frame}{Kontrollmatrix - Allgemein}
  \begin{itemize}
	  \item ist eine $(n-k) \times n$ Matrix
	  \item Werden von Generatormatrizen generiret
	  \item Berechnen "Syndrome" \space von empfangenen Nachrichten
	  \item Ermöglichen so die maximum likelyhood Dekodierung
  \end{itemize}
  \end{frame}
  
  \begin{frame}{Kontrollmatrix - Berechnung}
    Bestehen aus 2 Teilen
    \begin{itemize}
		\item Linker Teil: \newline Transponierte Nichtbasisspalten
		\item Rechter Teil: \newline Einheitsmatrix mit dem Wert "$-1$" \space als Diagonale
  	\end{itemize}
	(ein Beispiel wird anhand des Scriptes vorgeführt)
  \end{frame}
  
  
  \section{Syndromtabelle}
  \begin{frame}{Syndromtabelle - Allgemeines}
	  \begin{itemize}
	  \item Durch Syndrome ($(n - k)$ große Vektoren) können Codewortfehler errechnet werden
	  \item Jedes Syndrom beschreibt eine Nebenklasse
	  \item Codewortfehler ("Syndrome") werden mithilfe der Nebenklassenanführer erstellt
	  \item Nebenklassenanführer sind diejenigen einer Nebenklasse mit der kleinsten Hammingdistanz zum 0-Wort
	  \item Existieren mehrere Nebenklassenanführer ist keine eindeutige Zuordnung des Codewortes möglich
	  \end{itemize}
  \end{frame}
  
  \begin{frame}{Syndromtabelle - Erstellung}
	  2 Mögliche Wege
  	  \begin{itemize}
	  \item Transponiere die Kontrollmatrix und multipliziere jede mögliches Syndrom
	  \begin{itemize}
	  \item Vorteil: Sehr Performant und arbeitspeicher schonent
	  \end{itemize}
	  \item Erstelle alle möglichen Codeworter und deren Syndrome
	  \begin{itemize}
	  \item Finde minimale gewichte
	  \item Makiere Nebenklassen mit mehreren Nebenklassenanführeren
	  \item Vorteil: Wissen über mögliche, falsche Codewort Dekodierung
	  \end{itemize}
  	  \end{itemize}
  \end{frame}
    
   
   \section{Maximum-Likelhood Dekodierung}
   \begin{frame}{Maximum-Likelhood Dekodierung}
   \begin{itemize}
	\item Errechne Syndrom des erhaltenen Codewortes
	\item Ziehe den Nebenklassenanführer vom Codewort ab
	\item Überbleibendes Codewort ist das wahrscheinlichste
   \end{itemize}
   \end{frame}
  

\section*{Vielen Dank für die Aufmerksamkeit!}
  
\end{document}
